\documentclass[titlepage]{article}
\usepackage[french]{babel}
\usepackage[utf8]{inputenc}
\usepackage[T1]{fontenc}
\usepackage{amsmath}
\usepackage{listings} %code
\usepackage{graphicx} %images
\usepackage{float}
\usepackage[export]{adjustbox} %images
\usepackage{fullpage} %marges
\usepackage[usenames,dvipsnames]{xcolor} %couleurs
\usepackage{wrapfig}
\usepackage{lscape}
\usepackage{rotating}
\usepackage{fancyhdr}
\usepackage{hyphenat}
\usepackage{titlesec}
\usepackage{amsmath}
\usepackage{hyperref}
\usepackage{mathtools}
\DeclarePairedDelimiter\floor{\lfloor}{\rfloor}
\usepackage{array,multirow,makecell}
\setcellgapes{1pt}
\makegapedcells
\newcolumntype{R}[1]{>{\raggedleft\arraybackslash }b{#1}}
\newcolumntype{L}[1]{>{\raggedright\arraybackslash }b{#1}}
\newcolumntype{C}[1]{>{\centering\arraybackslash }b{#1}}
%\usepackage{listings} %code
%\usepackage[right=2cm]{geometry}%[left=2cm,right=2cm,top=2cm,bottom=2cm]
%\setlength{\voffset}{1pt}
\renewcommand{\headheight}{0.6in}
\setlength{\headwidth}{\textwidth}
\fancyhead[L]{}% empty left
\fancyhead[R]{ % right
   \includegraphics[height=0.53in]{logo_utc.jpg}
}
\setlength{\headsep}{0.2in}
\pagestyle{fancy}

\hyphenpenalty 10000

\newcommand{\refer}{
$\Rightarrow$
}

\newcommand{\deter}{
$\rightarrow$
}


\begin{document}%\sloppy
\title{SY09 Printemps 2017\\Compte-rendu TP1: Statistique descriptive, Analyse en composantes principales}
\author{Jiawei ZHU - Laëtitia BOUDEREAUX\\\\(GI04)}
\date{\today}
\maketitle

\section{Statistique descriptive}
\subsection{Notes}
\subsubsection{}
Le jeu de données contenu dans le fichier sy02-p2016.csv contient des informations relatives aux
étudiants inscrits à l’UV SY02 au semestre de printemps 2016. Dans le fichier, nous pouvons compter 296 données sur les étudiants. Chaque donnée décrit 12 attributs: nom de l'étudiant, sa spécialité (branche), son niveau (semestre d'étude), son statut, le dernier diplôme qu'il a obtenu, sa note de médian, le correcteur associé à sa note de médian, sa note de final, le correcteur associé à sa note de final, sa note total et son résultat.
\paragraph{}
Certaines données dans ce jeu peuvent être manquantes. Trois types de données manquantes existent:
\begin{itemize}
\item Dernier diplome obtenu dans le cas des étudiants étrangers
\item Notes du médian: si l'étudiant n'est pas venu au médian, il ne peut pas être noté
\item Notes du final: même cas que pour la note du médian. Cependant, on remarquera que les personnes qui ne sont pas venus au final ont eu une mauvaise note au médian.
\end{itemize}
\subsection{Données crabs}
\section{Conclusion}
\end{document}
